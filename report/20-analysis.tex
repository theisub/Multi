\chapter{Цель работы}

\textbf{Цель} данной работы - протестировать приложение чат-бот для отображения ближайших концертов музыкальных исполнителей из списка подписок.

\section{Описание тестируемой системы}

\textbf{Приложение} чат-бот состоит из следующих частей:
\begin{enumerate}
	\item сервер
	\item база данных
	\item клиентская часть
\end{enumerate}

\textbf{Сервер} написан на языке C\# с использованием API от ресурса KudaGo. Основное предназначение - обработка запросов от клиента.

\textbf{База данных} - реляционная СУБД MSSQL. Обращение к базе данных осуществляется с помощью Entity Framework. 

\textbf{Клиентская часть} представляет собой приложение Telegram, связь с клиентом осуществляется с помощью API Telegram.Bot.

Основные сущности базы данных:
\begin{enumerate}
	\item Users - таблица с пользователями
	\item Groups - таблица с музыкальными исполнителями
	\item Concerts - таблица с информацией о концертах
	\item Subscriptions - таблица с подписками пользователей
\end{enumerate}

\textbf{program.cs} - главный файл тестируемого приложения.

В файле \textbf{program.cs} поступают запросы от клиента, можно выделить следующие ключевые классы для обработки запросов:

\begin{enumerate}
	\item UserActions.cs - обработка запросов, связанных с информацией о пользователе
	\item GroupActions.cs - обработка запросов, связанных с информацией о музыкальных исполнителях
	\item ConcertActions.cs - обработка запросов, связанных с информацией о концертах
	\item CityActions.cs - обработка запросов, связанных с информацией о городе пользователя
	\item SubscriptionActions.cs - обработка запросов, связанных с информацией о подписках пользователей
\end{enumerate}

В клиентском приложении пользователь с помощью сообщения /start и после чего пользователь может использовать следующие команды:
\begin{enumerate}
	\item /city 'город пользователя' - добавление города пользователя в таблицу пользователей для дальнейшего поиска концертов в этом городе
	\item /add 'название исполнителя' - добавляет музыкального исполнителя в список подписок пользователя 
	\item /list - отображает подписки в виде пронумерованного списка
	\item /show - отображает ближайшие концерты исполнителей в городе пользователя из его подписок 
	\item /remove 'название исполнителя' - удаляет исполнителя из подписок пользователя
	\item /clear - очищает все подписки пользователя 
\end{enumerate}


\section{Рассматриваемые виды тестирования}

В данной работе будут рассмотрены следующие виды тестирования:
\begin{enumerate}
	\item модульное
	\item интеграционное
	\item регрессионное
	\item функциональное
	\item автоматизированное 
	
\end{enumerate}


\textbf{Интеграционное тестирование} предназначено для проверки связи между компонентами, а также взаимодействия с различными частями системы (операционной системой, оборудованием либо связи между различными системами).

\textbf{Регрессионное тестирование} - это вид тестирования направленный на проверку изменений, сделанных в приложении или окружающей среде (починка дефекта, слияние кода, миграция на другую операционную систему, базу данных, веб сервер или сервер приложения), для подтверждения того факта, что существующая ранее функциональность работает как и прежде.

\textbf{Функциональное тестирование} рассматривает заранее указанное поведение и основывается на анализе спецификаций функциональности компонента или системы в целом. 